\documentclass{article}

\usepackage{url,hyperref,lineno,microtype,subcaption}
\usepackage[onehalfspacing]{setspace}
\usepackage[tmargin=2.5cm,bmargin=2.5cm,lmargin=2.5cm,rmargin=2.5cm]{geometry}

\begin{document}

\large{\textbf{Micronekton Community Structure On The Southern Kerguelen Axis}}

Rowan Trebilco\textsuperscript{1},
Andrea Walters\textsuperscript{1,2},
Mark Hindell\textsuperscript{1,2},
Jess Melbourne-Thomas\textsuperscript{3},
Sophie Bestley\textsuperscript{4},
Martin Cox\textsuperscript{3},
Sven Gastauer\textsuperscript{1},
Andrew Constable\textsuperscript{3}


\textbf{Affiliations and addresses}
\tectsupescript{1} Antarcitc Climate & Ecosystems CRC, University of Tasmania, Hobart, rowan.trebilco@utas.edu.au\\
\tectsupescript{2} Institute of Marine and Antarctic Studies, University of Tasmania\\ 
\tectsupescript{3} Australian Antarctic Division, Kingston, Tasmania\\ 
\tectsupescript{4} CSIRO Oceans & Atmosphere, Hobart, Tasmania


\linenumbers
\section{Introduction}

The fish and macrozooplankton that inhabit the 

Mesopelagic as a black hole in understanding of ocean systems
Particularly true in the southern ocean

K-axis as a region of particular interest to Australia

This study: an overview of mesopelagic community structure

Previous studies have focused on distributions and associations of individual taxa and/or functional groups. 
While of great value for ... biogeography... 
Here we aim to provide a summary in a form that can directly inform ecosystem modelling 
Robust model representations will be important for guiding the future fisheries and conservation management in this area, and the strong biophysical gradients in the region make it an ideal testbed for model development

The aim of this study was to describe the summer composition and vertical distribution of the mesopelagic micronekton community and explore associations with biophysical...

We developed hypotheses that could explain the relationship between...


\section{Methods}

The mesopelagic community was sampled at 36 stations along the voyage track, from the surface to 1000 m, using an International Young Gadoid Pelagic Trawl net (IYGPT, with an opening of 188 m\textsuperscript{2}) equipped with a multiple opening and closing cod-end device (MIDOC). 
The MIDOC comprises 6 separate cod-ends (with a mesh size of 20 mm, terminating in a removable "soft" codend bag made of 0.5 mm mesh). 
The MIDOC allows cod-ends to be opened sequentially at pre-programmed intervals, such that each cod-end samples a different depth stratum.
The first cod end was open as the net descended from the surface to a maximum depth of 1000 m, then the remaining 5 cod-ends each sampled a 200 m depth band as the net returned to the surface (1000 – 800 m, 800 – 600 m, 600 – 400 m, 400 – 200 m, and 200 m – surface).
Nets were towed for 30 min at an average speed of 2.7 knots for each 200 m depth band (covering a mean distance of 1.35 nautical miles, and sweeping a mean volume of 450,800 m\textsuperscript{3}), and at 3.9 knots for 60 to 90 minutes for the first descending cod-end (covering a mean distance of 5.95 nautical miles and sweeping a mean volume of 1.98 x 10\textsuperscript{6} m\textsuperscript{3}). 


Catch was converted to densities by dividing numbers and weights by the volume swept for each cod end. Acoustic backscatter in the water column was characterised during tows using an Simrad EK60 echosounder operated at 38 kHz.
Acoustic data were filtered and quality controlled prior to the derivation of the total Nautical Area Scattering Coefficient (NASC) for the time period and depth range corresponding to each depth stratum. NASC is an acoustic density measure, corresponding to the acoustic energy per unit distance, which can be translated into biologically more meaningful biomass or abundance estimates, if the species composition and the sound scattering of an individual of the given species group is known.
TODO: say something more here


\section{Results}



\section{Discussion}

\emph{Previous work on biomass/abundance:}\\
BROKE W: only 332 fish and larvae and 58 squid collected from 125 target and routine RMTs at 60 stations \citep{Vandeputte:2010}\\
Hydrographic conditions and food availability have been identified as the major driving forces for E. antarctica to form concentrations (Loots et al 2007; Flores et al 2008)\\
Biomass density from night RMT8 and RMT25 hauls was 3.04g/1000m3 (Collins et al 2008). The main biomass of myctos and bathylagids was between 400 and 1000 m during the day and 0 - 400 m at night. \\
From RMT25 catches, density per m2 in stratum of 0-1000m has ranged from ~1.6 to 15 gm.m-2 (Collins et al 2008, Chindova 1987, Filin et al. 1990, Kozlov et al 1990)


\section{Acknowledgements}
Data presented here were collected on the RV Aurora Australis as part of the Kerguelen Axis ecosystem study. This study was carried out under the Australian Antarctic Science Program (AAS project 4344) with funding and logistical support from the Australian Antarctic Division, Department of Energy and Environment, and with support of the Antarctic Gateway Partnership (Project ID SR140300001) Australia and the Australian Governments Cooperative Research Centres Programme, through the Antarctic Climate and Ecosystems CRC. We thank the Science Technical Support Team of the Australian Antarctic Division and the master and crew of the Aurora Australis for their hard work and professionalism.  RT was supported by the RJL Hawke Postdoctoral Fellowship (AAS 4366).

\section{References}
\renewcommand\refname{}
\bibliographystyle{./bib/mee} 
\renewcommand{\section}[2]{}
\bibliography{./bib/biblio.bib}

\end{document}
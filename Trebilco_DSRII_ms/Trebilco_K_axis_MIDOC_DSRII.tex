\documentclass[12pt]{article}
\usepackage{setspace}
\usepackage{enumerate}
\usepackage{natbib}
\usepackage{textcomp}
\usepackage{hyperref}
\usepackage{lineno}
\usepackage{graphicx}
\usepackage[tmargin=2.5cm,bmargin=2.5cm,lmargin=2.5cm,rmargin=2.5cm]{geometry}

\begin{document}

{\setlength{\parindent}{0cm}
\pagenumbering{arabic} 
\linenumbers
\doublespacing
\thispagestyle{empty}


\centering \large{\textbf{Micronekton Community Structure \\on the Southern Kerguelen Axis}}

\normalsize
\flushleft

\textbf{Authors:}\\
Rowan Trebilco\textsuperscript{1*},
Andrea Walters\textsuperscript{1,2},
Mark Hindell\textsuperscript{1,2},
Jess Melbourne-Thomas\textsuperscript{3},
Sophie Bestley\textsuperscript{2},
Martin Cox\textsuperscript{3},
Sven Gastauer\textsuperscript{1},
Andrew Constable\textsuperscript{3}


\textsuperscript{1} Antarcitc Climate \& Ecosystems CRC, University of Tasmania, Australia\\
\textsuperscript{2} Institute of Marine and Antarctic Studies, University of Tasmania\\ 
\textsuperscript{3} Australian Antarctic Division, Kingston, Tasmania\\ 
% \textsuperscript{4} CSIRO Oceans \& Atmosphere, Hobart, Tasmania

\textsuperscript{*}Corresponding author:\\
Antarctic Climate and Ecosystems CRC\\
Private Bag 80\\
Hobart, TAS 7001\\
Australia\\
phone:+613 6226 7888\\
email:rowan.trebilco@utas.edu.au\\

\clearpage
\section{Abstract}
\linenumbers

A goal of this voyage was to characterise pelagic foodweb structure and major energy pathways in the region, with a strong focus on the mesopelagic community, and to pilot methodologies for future ecosystem observation and monitoring

\clearpage
\section{Introduction}


The fishes, cephalopods, crustaceans, salps, cnidarians and other macrozooplankton that inhabit the upper 1000 m of the open oceans (hereafter collectively termed micronekton) represent a key area of uncertainty in our understanding of the structure and function of marine ecosystems worldwide \citep{StJohn:2016cr,Young:2015}.
These mid trophic-level groups support the passage of energy and biomass from primary producers to large consumers at higher trophic-levels (including marine mammals, seabirds, and commercially important fishes), and they collectively dominate the total abundance and biomass of complex metazoan life in the ocean \cite{BarOn:2018bj,Irigoien:2014}.
Despite increasing recognition of the importance of micronekton, major gaps remain in our understanding of the ecology of these groups; notably regarding their distribution, abundance, biomass and trophodynamics \cite{Young:2015,Newman:SOOS_CWP,Davison:2015fq}. 
This is in large part because of the difficulties of sampling and observing the mid-water zone, but also because observations have been patchy in space and time \cite{Kaartvedt:2012ji,Newman:SOOS_CWP}.

In the Southern Ocean, micronekton support wildlife populations including whales, penguins, seals, and seabirds, valuable fisheries (both directly in the case of krill, and indirectly as the main prey of toothfish) ... important role in carbon export

Attention has focused on top 200 m
Lack of information regarding both the distribution of mesopelagic or

These challenges for 

Previous studies of micronekton have mainly focused

The 
K-axis as a region of particular interest to Australia

This study: an overview of mesopelagic community structure

Previous studies have focused on distributions and associations of individual taxa and/or functional groups. 
While of great value for ... biogeography... 
Here we aim to provide a summary in a form that can directly inform ecosystem modelling 
Robust model representations will be important for guiding the future fisheries and conservation management in this area, and the strong biophysical gradients in the region make it an ideal testbed for model development

The overall aim of the Kerguelen Axis study
The specific aims of this manuscript are to: 
	(1) provide an overview of the composition of micronekton catch from IYGPT/MIDOC mid-water trawls and how;
	(2) examine how local oceanographic conditions predict differences in catch composition among sampling stations; and
	(3) examine the relationship between total acoustic backscatter and catch composition.
% summer composition and vertical distribution of the mesopelagic micronekton community and explore associations with biophysical...
% We developed hypotheses that could explain the relationship between...
\emph{TODO}: decide whether to include acoustics: delete (3 here if not)


More detailed examination of taxon specific distributions, trophic relationships, and environmental associations are provided in other manuscripts in this issue (e.g.
fish -- Woods, Riaz, Walters;  
Macrozooplankton -- Weldrick, Clark, ??others
)
and elsewhere (e.g.
Kerguelen plateau symposium chapters -- Clark, Trebilco, Woods
)


\section{Methods}



The mesopelagic community was sampled at 36 stations along the voyage track, from the surface to 1000 m, using an International Young Gadoid Pelagic Trawl net (IYGPT, with an opening of 188 m\textsuperscript{2}) equipped with a multiple opening and closing cod-end device (MIDOC). 
The MIDOC comprises 6 separate cod-ends (with a mesh size of 20 mm, terminating in a removable "soft" codend bag made of 0.5 mm mesh). 
The MIDOC allows cod-ends to be opened sequentially at pre-programmed intervals, such that each cod-end samples a different depth stratum.
The first cod end was open as the net descended from the surface to a maximum depth of 1000 m, then the remaining 5 cod-ends each sampled a 200 m depth band as the net returned to the surface (1000 – 800 m, 800 – 600 m, 600 – 400 m, 400 – 200 m, and 200 m – surface).
Nets were towed for 30 min at an average speed of 2.7 knots for each 200 m depth band (covering a mean distance of 1.35 nautical miles, and sweeping a mean volume of 450,800 m\textsuperscript{3}), and at 3.9 knots for 60 to 90 minutes for the first descending cod-end (covering a mean distance of 5.95 nautical miles and sweeping a mean volume of 1.98 x 10\textsuperscript{6} m\textsuperscript{3}). 


Catch was converted to densities by dividing numbers and weights by the volume swept for each cod end. Acoustic backscatter in the water column was characterised during tows using an Simrad EK60 echosounder operated at 38 kHz.
Acoustic data were filtered and quality controlled prior to the derivation of the total Nautical Area Scattering Coefficient (NASC) for the time period and depth range corresponding to each depth stratum. NASC is an acoustic density measure, corresponding to the acoustic energy per unit distance, which can be translated into biologically more meaningful biomass or abundance estimates, if the species composition and the sound scattering of an individual of the given species group is known.
TODO: say something more here


\section{Results}

Results fig 1: bubble plots of catch per station
	TODO: add SB oceanographic zones

Results fig 2: ,

\section{Discussion}

\emph{Previous work on biomass/abundance:}\\
BROKE W: only 332 fish and larvae and 58 squid collected from 125 target and routine RMTs at 60 stations \citep{Vandeputte:2010}\\
Hydrographic conditions and food availability have been identified as the major driving forces for E. antarctica to form concentrations (Loots et al 2007; Flores et al 2008)\\
Biomass density from night RMT8 and RMT25 hauls was 3.04g/1000m3 (Collins et al 2008). The main biomass of myctos and bathylagids was between 400 and 1000 m during the day and 0 - 400 m at night. \\
From RMT25 catches, density per m2 in stratum of 0-1000m has ranged from ~1.6 to 15 gm.m-2 (Collins et al 2008, Chindova 1987, Filin et al. 1990, Kozlov et al 1990)


- particularly fishy stations: 15, 23,27 (28 deep, 3 shallow)

- big krill site was Midoc 8. 275 kg of krill all in CE1. Total swept volume for all 6 cod ends at this site was 4466791; for density of 0.062 g/m3.

Collins 2012:
"Bathylagids were patchily distributed, but were abundant in the lower mesopelagic zone (4400 m) and are potentially significant zooplankton consumers"
"The ecological role of the bathylagids is poorly known but, given the abundance of this family, studies of their role as both predator and prey should be a high priority."

\section{Acknowledgements}
Data presented here were collected on the RV Aurora Australis as part of the Kerguelen Axis ecosystem study. This study was carried out under the Australian Antarctic Science Program (AAS project 4344) with funding and logistical support from the Australian Antarctic Division, Department of Energy and Environment, and with support of the Antarctic Gateway Partnership (Project ID SR140300001) Australia and the Australian Governments Cooperative Research Centres Programme, through the Antarctic Climate and Ecosystems CRC. We thank the Science Technical Support Team of the Australian Antarctic Division and the master and crew of the Aurora Australis for their hard work and professionalism.  RT was supported by the RJL Hawke Postdoctoral Fellowship (AAS 4366).

\section{References}
% \renewcommand\refname{}
\bibliographystyle{mee} 
% \renewcommand{\section}[2]{}
\bibliography{jshort,KAX_biblio}

\end{document}